%
% Academic CV LaTeX Template
% Author: Dubasi Pavan Kumar
% LinkedIn: https://www.linkedin.com/in/im-pavankumar/
% License: MIT
%
% For errors, suggestions, or improvements, please contact:
% Email: pavankumard.pg19.ma@nitp.ac.in

\documentclass{article}
\usepackage[letterpaper]{geometry}

% % Package imports
\usepackage{latexsym}
\usepackage{xcolor}
\usepackage{float}
\usepackage{ragged2e}
\usepackage[empty]{fullpage}
\usepackage{wrapfig}
\usepackage{lipsum}
\usepackage{tabularx}
\usepackage{titlesec}
\usepackage{geometry}
\usepackage{marvosym}
\usepackage{verbatim}
\usepackage{enumitem}
\usepackage{fancyhdr}
\usepackage{multicol}
\usepackage{graphicx}
\usepackage{cfr-lm}
\usepackage[T1]{fontenc}
\usepackage{fontawesome5}
\usepackage{xfp} % for float calculations

% Color definitions
\definecolor{darkblue}{RGB}{0,0,139}

% Page layout
\setlength{\multicolsep}{0pt} 
\pagestyle{fancy}
\fancyhf{} % clear all header and footer fields
\fancyfoot{}
\renewcommand{\headrulewidth}{0pt}
\renewcommand{\footrulewidth}{0pt}
\geometry{left=1.2cm, top=1.0cm, right=1.2cm, bottom=1.2cm}
\setlength{\footskip}{5pt} % Addressing fancyhdr warning

% Hyperlink setup (moved after fancyhdr to address warning)
\usepackage[hidelinks]{hyperref}
\hypersetup{
    colorlinks=false,
    linkcolor=darkblue,
    filecolor=darkblue,
    urlcolor=darkblue,
}

% Custom box settings
\usepackage[most]{tcolorbox}
\tcbset{
    frame code={},
    center title,
    left=0pt,
    right=0pt,
    top=0pt,
    bottom=0pt,
    colback=gray!20,
    colframe=white,
    width=\dimexpr\textwidth\relax,
    enlarge left by=-2mm,
    arc=0pt,outer arc=0pt,
}

% URL style
\urlstyle{same}

% Text alignment
\raggedright
\setlength{\tabcolsep}{0in}


% as this increases, the gaps should increase 
\newcommand{\baseSpace}{4.0}

\titleformat{\section}
{\vspace{\fpeval{1.5 * \baseSpace}pt}\scshape\raggedright\large}
  {}{0em}
  {\sectiontitlerule}
  []

% overwrite default section spacing 
\titlespacing{\section}{0pt}{0pt}{0pt}
% \setlist[itemize]{topsep=0pt}
\setlist[itemize]{nosep}

% \newcommand{\sectiontitlerule}[1]{#1 \hspace{0.5mm}\color{black}\titlerule \vspace{-2.5pt}}
\newcommand{\sectiontitlerule}[1]{#1 \vspace{-3.6mm} \newline \color{black}\titlerule \vspace{-2mm}}

% Custom commands
\newcommand{\resumeItem}[2]{
  \item{
	\textbf{#1}{\hspace{0.5mm}#2 \vspace{\fpeval{0 * \baseSpace} pt}}
  }
}

\newcommand{\resumePOR}[3]{
	\vspace{\fpeval{0 * \baseSpace} mm}\item
    \begin{tabular*}{0.97\textwidth}[t]{l@{\extracolsep{\fill}}r}
        \textbf{#1}\hspace{0.3mm}#2 & \textit{\small{#3}} 
    \end{tabular*}
	\vspace{\fpeval{0 * \baseSpace} mm}
}

\newcommand{\resumeSubheading}[4]{
	\vspace{\fpeval{1.5 * \baseSpace} pt}
    \begin{tabular*}{0.99\textwidth}[t]{l@{\extracolsep{\fill}}r}
		\textbf{\normalsize{#1}} & \textit{\normalsize{#4}} \\
        \textit{\normalsize{#3}} &  \normalsize{#2}\\
    \end{tabular*}
	\vspace{\fpeval{0.5 * \baseSpace} pt}
}

\newcommand{\resumeProject}[2]{
	\vspace{\fpeval{1.5 * \baseSpace} pt}
    \begin{tabular*}{0.99\textwidth}[t]{l@{\extracolsep{\fill}}r}
        \textbf{#1} & \textit{\normalsize{#2}}
        % \normalsize{\textit{#2}} & \normalsize{#4}
    \end{tabular*}
	\vspace{\fpeval{0.5 * \baseSpace} pt}
}

\newcommand{\resumeResearch}[4]{
	\vspace{\fpeval{1.5 * \baseSpace} pt}
    \begin{tabular*}{0.99\textwidth}[t]{l@{\extracolsep{\fill}}r}
		\textbf{\normalsize{#1}} & \textit{\normalsize{#4}} \\
        \textit{\normalsize{#3}} &  \normalsize{#2}\\
    \end{tabular*}
	\vspace{\fpeval{0.5 * \baseSpace} pt}
}

\newcommand{\resumeAward}[2]{
	\vspace{\fpeval{1.5 * \baseSpace} pt}
    \begin{tabular*}{0.99\textwidth}[t]{l@{\extracolsep{\fill}}r}
        \textbf{#1} & \textit{\normalsize{#2}} \\
    \end{tabular*}
	\vspace{\fpeval{0.0 * \baseSpace} pt}
}

\newcommand{\resumeSubItem}[2]{\resumeItem{#1}{#2}\vspace{\fpeval{0 * \baseSpace} pt}}

\renewcommand{\labelitemi}{$\vcenter{\hbox{\tiny$\bullet$}}$}
\renewcommand{\labelitemii}{$\vcenter{\hbox{\tiny$\circ$}}$}

\newcommand{\resumeSubHeadingListStart}{}
\newcommand{\resumeHeadingSkillStart}{\vspace{\fpeval{1.7 * \baseSpace} pt}\begin{itemize}[leftmargin=*,itemsep=\fpeval{1.0 * \baseSpace} pt, rightmargin=2ex]}
\newcommand{\resumeItemListStart}{\begin{itemize}[leftmargin=*,labelsep=1mm,itemsep=\fpeval{0.8 * \baseSpace} pt]\normalsize}

\newcommand{\resumeSubHeadingListEnd}{}
\newcommand{\resumeHeadingSkillEnd}{\end{itemize}\vspace{\fpeval{0 * \baseSpace} pt}}
\newcommand{\resumeItemListEnd}{\end{itemize}}

\newcolumntype{L}{>{\raggedright\arraybackslash}X}%
\newcolumntype{R}{>{\raggedleft\arraybackslash}X}%
\newcolumntype{C}{>{\centering\arraybackslash}X}%

% Commands for icon sizing and positioning
\newcommand{\socialicon}[1]{\raisebox{-0.05em}{\resizebox{!}{1em}{#1}}}

% Font options
\newcommand{\headerfonti}{\fontfamily{phv}\selectfont} % Helvetica-like (similar to Arial/Calibri)
\newcommand{\headerfontii}{\fontfamily{ptm}\selectfont} % Times-like (similar to Times New Roman)
\newcommand{\headerfontiii}{\fontfamily{ppl}\selectfont} % Palatino (elegant serif)


\linespread{1.00}

\begin{document}
\headerfontiii

% Header
\begin{center}
	{\huge\textbf{Kristofer Fannar Bjornsson}}
\end{center}
\vspace{-20pt}

\begin{center}
	\small{
		(646) 229 0496 |
		\href{mailto:kristofer.bjornsson@columbia.edu}{kristofer.bjornsson@columbia.edu} |
		\href{https://kristoferfannar.github.io}{kristoferfannar.github.io} |
		\socialicon{\faLinkedin} \href{https://www.linkedin.com/in/kristoferfannar/}{kristoferfannar} |
		\socialicon{\faGithub} \href{https://github.com/kristoferfannar}{kristoferfannar}
	}
\end{center}
\vspace{-7pt}

% \begin{center}
%     \small{Your City, Your State - Your ZIP, Your Country}
% \end{center}
%
% \vspace{-4mm}
%
% \section{\textbf{Objective}}
% \vspace{1mm}
% Seeking an opportunity to contribute as a software engineering intern for summer 2025. Available for 12 weeks starting May 19th.
% My interests lie in the fields of systems \& cybersecurity as well as solving difficult, fun problems.
% \vspace{-2mm}


\section{\textbf{Education}}
\resumeSubHeadingListStart

\resumeSubheading
{Columbia University}{New York, NY}
{MSc Computer Science | Software Systems track}{Aug 2024 - Dec 2025 (Expected)}
\resumeItemListStart
% \item 4.0 GPA $\cdot$ ICPC 2024 GNY Regionals participant
% \item 4.0 GPA $\cdot$ Blockchain@Columbia member % $\cdot$ ICPC 2024 GNY Regionals participant
% \item 4.0 GPA $\cdot$ ICPC 2024 GNY Regionals participant
\item ICPC 2024 GNY Regionals participant
\newline
\begin{tabular}{l}
	\textit{Selected Coursework} | 
	Operating Systems $\cdot$ 
	% Programming Languages \& Translators
	% Design Using C++ $\cdot$
	Cryptography $\cdot$ 
	% Competitive Programming $\cdot$
	Programming \& Problem Solving
\end{tabular}
\resumeItemListEnd

\resumeSubheading
{Reykjavik University}{Reykjavik, IS}
{BSc Software Engineering}{Aug 2021 - May 2024}
\resumeItemListStart
\item First Class with Distinction $\cdot$ Dean's list F22 \& F23 $\cdot$ Freshman grant F21
% \item Teaching Assistant for Algorithms F23 $\cdot$ Web Programming S24 $\cdot$ Computer Networks F24
\begin{tabular}{l}
	\textit{Selected Coursework} | 
	% Mobile App Development $\cdot$
	Cybersecurity $\cdot$ 
	Concurrent \& Distributed Programming $\cdot$ 
	Computer Networks
\end{tabular}
% \begin{tabular*}{0.98\textwidth}[t]{l@{\extracolsep{\fill}}r}
% 	\textit{Exchange semester @ University of Copenhagen} &  \textit{Spring 2023}
% \end{tabular*}
\resumeItemListEnd

\section{\textbf{Work Experience}}
\resumeSubHeadingListStart

\resumeSubheading
{\href{https://bloomberg.com}{Bloomberg}}{New York, NY}
{Security Engineer Intern}{June 2025 - Aug 2025}
\resumeItemListStart
\item Built a CLI tool in Python to automate exploit testing on >2.000 CVE vulnerabilities on >50.000 Bloomberg machines
\item Helped prove exploitability on over 700 critical \& high vulnerabilites in Bloomberg environment
% \item Parsed exploits for >2.000 CVEs from metasploit, finding vulnerabilites within the entire Bloomberg environment, totalling >50.000 services
% \item Researched Linux Kernel vulnerabilites, working around memory corrupting exploits to produce crash-safe proof of concepts, safe for testing on production systems
\resumeItemListEnd

\resumeSubheading
{\href{https://keystrike.com}{Keystrike (Startup)}}{Reykjavik, IS}
{Software Engineer Intern}{May 2024 - Jan 2025}
\resumeItemListStart
\item Migrated security desktop driver from Windows to macOS in Rust, now adopted by 1/3rd of company's userbase

% \item Performed security assessment for \& built same driver functionality into secure Chrome Extension using pure HTML, Javascript and Tailwind generated CSS - dependency free

\item Wrote proof of concept for and resolved critical internal vulnerability, preventing injected inputs from being validated
% \item Created custom Rust bindings for communicating with macOS API, prototyping in Swift \& Objective C
% \item Main developer through entire SDLC, from non macOS compiling Windows coupled codebase to external beta testing

% \item Built using high security standards for storing secrets in Apple Keychain, preventing injected inputs \& signing input attestations

\item Setup extensive test suite running on automatic app building \& deployment pipelines for seamless distribution through GitHub Actions
\resumeItemListEnd

\resumeSubheading
{\href{https://www.ru.is}{Reykjavik University}}{Reykjavik, IS}
{Teaching Assistant | Algorithms $\cdot$ Web Programming $\cdot$ Computer Networks}{Aug 2023 - Dec 2024}
\resumeItemListStart
\item Held TA sessions, mentoring 60+ students through solving problem sets over 3 semesters, teaching time complexity, searching, sorting etc. in Java (Algorithms), and HTML, CSS \& Javascript (Web Programming)
% \item Evaluated student submissions \& provided feedback to Professors on assignments and exams
\item Created bash script to validate HTML structure for code submissions, saving > 400 minutes of grading per assignment
\resumeItemListEnd


\resumeSubheading
{\href{https://www.syndis.is}{Syndis}}{Reykjavik, IS}
% {Software Engineer Intern}{May 2023 - Jan 2024}
{Security Engineer Intern}{May 2023 - Jan 2024}
\resumeItemListStart
\item Collaborated in a team of engineers to design and implement a full-stack web app for Incident Response handling for Business Continuity Planning with React, Express \& PostgreSQL through Docker
% \item Implemented secure in-house authentication service enabling multi-role privileges using JWT tokens 
% \item Conducted interviews with 5+ prospective clients to streamline Business Continuity Planning onboarding process from 2 weeks to 1 hour, creating an automatic incident response plan and customizable playbooks
\resumeItemListEnd

% \resumeSubheading
% {\href{https://hafogvatn.is/en}{Marine \& Freshwater Research Institute}}{Reykjavik, IS}
% {Software Engineer Intern}{May 2022 - Jan 2023}
% \resumeItemListStart
% \item Created web service using Java's Spring framework to collect and persist >10.000 daily events into local Oracle SQL server from fish passing through graders in Icelandic fisheries, used to improve fishing quota estimates
% % \item Programmed API to receive data through multiple grader protocols, creating private endpoints for fetching data
% \item Prototyped a method for estimating fish origin probabilities based off of catch date, vessel location and amount fished
% \resumeItemListEnd


% \resumeSubheading
% {Commercial College of Iceland}{Reykjavik, IS}
% {Secondary | Physics}{Aug 2018 - May 2021}
% \resumeItemListStart
% \item GPA: 9.2/10
% \resumeItemListEnd

\resumeSubHeadingListEnd

\section{\textbf{Research Experience}}
\resumeSubHeadingListStart

\resumeResearch
{Columbia University}{New York, NY}
{Research Assistant | Software Systems Lab}{Jan 2025 - Present}
\resumeItemListStart
\item Optimized software systems for satellites using eBPF, submitting a paper to HotNets '25 
% \item Added support for JIT compilation on uBPF virtual machine for RISC-V architecture
\item Added floating point arithmetic support on eBPF JIT compiler in C++ using LLVM, parsing new eBPF VM instructions and modifying the compiler to generate native floating point IR
\resumeItemListEnd

% \resumeResearch
% {Reykjavik University \& Keystrike}
% {Reykjavik, IS}
% {\href{https://skemman.is/bitstream/1946/47651/1/Reykjavik_University_Keystrike_Final_Report.pdf}{Thesis Researcher}}
% {Oct 2023 - May 2024}
% % {Skills: String comparison $\cdot$ Algorithmic research \& evaluation $\cdot$ Go $\cdot$ Python $\cdot$ Protobuf $\cdot$ Cryptography}
% % {\href{https://skemman.is/bitstream/1946/47651/1/Reykjavik_University_Keystrike_Final_Report.pdf}{report \footnotesize{\faFile}}}
% \resumeItemListStart
% % \item Researched ways to expand Keystrike's main product, then known as the Sanctum Guard, to the web
% \item Developed method for continuous authentication \& message verification over web protocols 
% \item Evaluated existing \& developed novel string comparison algorithms to determine whether a sequence of typed inputs could produce an edited string, reaching over $\ge 80\%$ confidence on average
% \item Rewrote Keystrike's proprietary cryptographic attestation protocol in Python with ECDSA signatures, wrapped into a FastAPI server
% \resumeItemListEnd

\resumeSubHeadingListEnd

\section{\textbf{Projects}}
\resumeSubHeadingListStart

\resumeProject
{\href{https://github.com/kristoferfannar/premking}{PremKing}}
% {Skills: React Native $\cdot$ Go $\cdot$ AWS EC2 $\cdot$ OAuth $\cdot$ Postgres $\cdot$ App Store \& Testflight $\cdot$ Lets Encrypt $\cdot$ Nginx}
{\href{https://github.com/kristoferfannar/premking}{github.com/kristoferfannar/premking \footnotesize{\faGithub}}}
\resumeItemListStart
\item Launched full-stack mobile app on \href{https://apps.apple.com/is/app/premking/id6476774713}{the App Store}, managing >100 weekly users, using React Native, Golang backend REST API deployed on AWS, connected to PostgresSQL database, with push notifications and OAuth
\item Utilized cronjobs to fetch data from external API, updating user scores \& calculating positions on new responses
\resumeItemListEnd

% \resumeProject
% {\href{https://github.com/lmkdatabase/lmkdb}{lmkDB}}
% {\href{https://github.com/lmkdatabase/lmkdb}{github.com/lmkdatabase \footnotesize{\faGithub}}}
% \resumeItemListStart
% 	\item Developed GPU accelerated SQL-like database from scratch using modern C++ and CUDA, calculating JOIN operations on 10.000.000+ rows $\approx$ 10x faster than on non-GPU machines
% \resumeItemListEnd

% \resumeProject
% {Shadow Page Table | Linux Kernel}
% {}
% \resumeItemListStart
% \item Added Linux Kernel system call for tracking \& monitoring specified process's virtual memory area ranges
% \item Monitored page table accesses in kernel, continuously updating a custom shared "shadow" page table, writable by kernel and readable from user space through \emph{remap\_pfn\_range}
% \resumeItemListEnd


\resumeProject
{Oven Scheduler | Linux Kernel}
{}
\resumeItemListStart
\item Created new default process scheduler, called "Oven", with C in the Linux Kernel, beating CFS (current default scheduler) by $\approx$ 17\% on single core and $\approx$ 5\% on 4 cores on selected benchmarks
	% \item Performed benchmarks on modified Linux Kernel using eBPF through VMWare virtual machine
\resumeItemListEnd

% \resumeProject
% {Fly | Programming Language}
% {}
% \resumeItemListStart
% \item Implemented functional programming language using OCaml, building a compiler frontend to LLVM IR
% \resumeItemListEnd

% \resumeProject
% {Ladebug | Debugger}
% {}
% \resumeItemListStart
% \item Built a debugger for ELF executables in Linux with \texttt{ptrace}, implementing breakpoints with INT3 and register inspection 
% \resumeItemListEnd

% \resumeProject
% {\href{https://github.com/lsig/raft}{Raft}}
% % {Skills: Concurrent \& Distributed Programming $\cdot$ Go $\cdot$ Mutual Exclusion $\cdot$ Channels $\cdot$ Consensus $\cdot$ Protobuf}
% {\href{https://github.com/lsig/raft}{github.com/lsig/raft \footnotesize{\faGithub}}}
% \resumeItemListStart
% \item Recreated \href{https://raft.github.io/}{Raft Consensus algorithm} from scratch in Golang over gRPC, using concurrent nodes in distributed network
% \item Consistently replicated logs between 10+ servers bouncing over $100.000$ concurrent messages on a network
% \resumeItemListEnd


% \resumeProject 
% {\href{https://github.com/DesktopDefender/DesktopDefender}{Desktop Defender}}
% % {Skills: \href{https://tauri.app}{Tauri}, Rust, Nextjs, LAN, Router, mDNS, Cyber Security, Encryption, Password cracking}
% {\href{https://github.com/DesktopDefender/DesktopDefender}{github.com/desktopdefender \footnotesize{\faGithub}}}
% \resumeItemListStart
% \item Built a network security desktop application in Tauri, combining NextJS with Tailwind \& TypeScript with Rust, for monitoring devices \& traffic on home networks
% \item Brute forced admin credentials on home network's router by scraping unauthorized endpoints on admin portal
% % \item Found network devices through ARP, built multithreaded Rust port scanner, scanning router \& devices for open ports
% \resumeItemListEnd

% \resumeProject
% {RC Car}
% {}
% \resumeItemListStart
% \item Soldered and developed a remote controlled RC car using CircuitPython on a microcontroller with collision avoidance, path following and convoying
% \resumeItemListEnd

\resumeSubHeadingListEnd

\section{\textbf{Technical Skills}}
\resumeHeadingSkillStart
\resumeSubItem{Languages| }
{
	% Java (Spring) $\cdot$ 
	% Golang $\cdot$
	% (HTML $\cdot$ CSS $\cdot$ JS) $\cdot$
	C $\cdot$ C++ $\cdot$
	% Typescript $\cdot$
	% Javascript $\cdot$
	Node $\cdot$
	Python $\cdot$
	React $\cdot$ % NextJS $\cdot$
	Rust  $\cdot$
	PostgreSQL % $\cdot$ 
	% Java $\cdot$
	% Bash $\cdot$
}

\resumeSubItem{Other| }
{
	% App Development $\cdot$
	Linux Kernel $\cdot$
	AWS $\cdot$
	% LLVM $\cdot$
	CI/CD $\cdot$
	eBPF $\cdot$
	% Tailwind $\cdot$ 
	% FastAPI $\cdot$
	% GitHub $\cdot$ 
	% Figma $\cdot$  
	% Jest $\cdot$
	% Apple Codesigning $\cdot$ 
	% macOS API $\cdot$ 
	% WireShark $\cdot$ 
	Git $\cdot$ 
	Docker % $\cdot$
	% Kubernetes % $\cdot$ 
	% REST $\cdot$ 
	% gRPC $\cdot$ 
	% TCP/IP
}

% \resumeSubItem{Interests| }
% {
% 	Distributed Systems $\cdot$
% 	Cybersecurity $\cdot$
% 	% Open Source $\cdot$ 
% 	% Problem Solving $\cdot$
% 	% SDLC $\cdot$
% 	Operating Systems % $\cdot$ 
% 	% Cryptocurrencies $\cdot$ 
% 	% Competitive Programming $\cdot$ 
% 	% CTFs
% }

\resumeHeadingSkillEnd

% \section{\textbf{Awards}}
% \resumeSubHeadingListStart
%
% \resumeAward
% {Icelandic National Gymnasium (11th - 13th grade) Math Competition}
% % {Icelandic Mathematical Society (ice. Íslenska Stærðfræðifélagið)}
% {2nd 2018 {}[\href{https://www.stae.is/stak/keppnin2018}{\textcolor{black}{\faIcon{globe}}}] | \textit{16th - 17th 2019 {}[\href{https://www.stae.is/stak/keppnin2019}{\textcolor{black}{\faIcon{globe}}}]}}
%
% \resumeAward
% {Icelandic National Compulsory School (1st - 10th grade) Math Competition}
% % {(ice. Stærðfræðikeppni Grunnskólanema)}
% {\textit{4th 2016 \href{http://www.lagafellsskoli.is/forsida/frettir/frett/2016/03/17/Staerdfraedikeppni-grunnskolanna-a-elsta-stigi/}{[\faGlobe]}}| \textit{4th 2017} | \textit{4th 2018 \href{http://www.lagafellsskoli.is/forsida/frettir/frett/2018/04/10/Urslit-staerdfraedikeppni-grunnskolanna-2018/}{[\faGlobe]}}}
%
% \resumeSubHeadingListEnd

\end{document}
